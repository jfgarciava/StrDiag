\documentclass[]{standalone}
\usepackage[utf8]{inputenc}
\usepackage{amsmath}
\usepackage{tikz}
\usetikzlibrary{cd}
\usetikzlibrary{arrows,arrows.meta}

\newcommand{\fComp}{} %\otimes
\newcommand{\nComp}{\circ}
\newcommand{\at}{}
\newcommand{\mc}[1]{\mathcal{#1}}
\newcommand{\id}[1]{\mathrm{id}_{#1}}
\newcommand{\Id}[1]{\mathrm{Id}_{#1}}
\newcommand{\applyNt}[2]{{#1}_{#2}}
\newcommand{\obj}[2]{{#1} \in {#2}}
\newcommand{\fc}[3]{{#1}: {#3} \leftarrow {#2}}
\newcommand{\nt}[3]{{#1}:\mathop{\Downarrow}^{#2}_{#3}}
\newcommand{\morf}[3]{{#1}: {#2} \rightarrow {#3}}


\begin{document}
	
	
	
Cada diagrama describe una 2-celda

$\displaystyle
Diagrama de tipo Nt
(@TeminalCat@)--A-->(C)--T-->(C)
 ||
r
 ||
\ /
(@TeminalCat@)--X-->(E)
$

que rellena un diagrama conmutativo

\begin{tikzcd}
%(GR2)
\end{tikzcd}

\begin{tikzpicture}[]
%(STR)
\end{tikzpicture}
\end{document}